\documentclass{article}
\usepackage[UTF8]{ctex}
\usepackage{geometry}
\usepackage{multirow}
\usepackage{natbib}
\geometry{left=3.18cm,right=3.18cm,top=2.54cm,bottom=2.54cm}
\usepackage{graphicx}
\pagestyle{plain}	
\usepackage{setspace}
\usepackage{enumerate}
\usepackage{caption2}
\usepackage{datetime} %日期
\renewcommand{\today}{\number\year 年 \number\month 月 \number\day 日}
\renewcommand{\captionlabelfont}{\small}
\renewcommand{\captionfont}{\small}
\begin{document}

\begin{figure}
    \centering
    \includegraphics[width=8cm]{upc.png}

    \label{figupc}
\end{figure}

	\begin{center}
		\quad \\
		\quad \\
		\heiti \fontsize{45}{17} \quad \quad \quad 
		\vskip 1.5cm
		\heiti \zihao{2} 《计算科学导论》个人职业规划
	\end{center}
	\vskip 2.0cm
		
	\begin{quotation}
% 	\begin{center}
		\doublespacing
		
        \zihao{4}\par\setlength\parindent{7em}
		\quad 

		学生姓名:\underline{\qquad  马青 \qquad \qquad}

		学\hspace{0.61cm} 号:\underline{\qquad 1907010223\qquad}
		
		专业班级:\underline{\qquad 计科1902 \qquad  }
		
        学\hspace{0.61cm} 院:\underline{计算机科学与技术学院}
% 	\end{center}
		\vskip 1.5cm
		\centering
		\begin{table}[h]
            \centering 
            \zihao{4}
            \begin{tabular}{|c|c|c|c|c|c|c|c|c|}
            % 这里的rl 与表格对应可以看到,姓名是r,右对齐的;学号是l,左对齐的;若想居中,使用c关键字。
                \hline
                \multicolumn{5}{|c|}{分项评价} &\multicolumn{2}{c|}{整体评价}  & 总    分 & 评 阅 教 师\\
                \hline
                自我 & 环境 & 职业 & 实施 & 评估与 & 完整性 & 可行性 &\multirow{2}*{} &\multirow{2}*{}\\
                分析& 分析& 定位 & 方案 & 调整 & 20\% & 20\% & ~&~ \\\            
                10\% & 10\% & 15\% & 15\% & 10\% & &  &~ &~\\
                \cline{1-7} 
                & & & & & & & ~&~ \\
                & & & & & & & ~&~ \\
                \hline      
            \end{tabular}
        \end{table}
		\vskip 2cm
		\today
	\end{quotation}

\thispagestyle{empty}
\newpage
\setcounter{page}{1}
% 在这之前是封面,在这之后是正文
\section{自我分析}
	\par
\subsection{自然条件}
性别:男。\par
年龄:18.\par
身体条件:身体比较壮,平时积极锻炼,注意规律的作息生活,饮食也是注意荤素搭配,但体育没拿满分,所以还是要多多强身健体。\par
健康情况:身体健康,无重大疾病。\par
学习城市:青岛。\par
家乡城市:信阳。\par
\subsection{性格分析}
我是一个比较乐观,生活态度积极的人,虽然也常对生活中一些不顺意之事难以看开,发出抱怨之词,但我也会时刻提醒自己,抱怨仅作舒缓情绪,要想改变现状,还要积极提升自己的能力。\par
而对于一些事情,我有时会表现出急躁的情绪,难以控制自己的言辞,但我识时务的能力往往会让我看清现实,即使满心不爽,也会把事情做完。\par
对待朋友,我认为自己也是真诚的,不会去损害朋友利益来让自己获利,有好处也会和朋友说,算是有福同享了。\par
我对于新奇的事物往往也充满着兴趣,乐于参与、探索。\par
\subsection{教育与学习经历}
小学:平顶山市新程街小学,担任过组长、课代表之职。\par
初中:平顶山市第四十三中学,担任过课代表之职。\par
高中:平顶山市一中,担任过课代表。\par
大学:中国石油大学(华东)。\par
\subsection{工作与社会阅历}
因家中经济条件比较差,所以从小就帮助父母做生意,卖过烧鸡等熟食,前些年政府对烟花售卖还没有禁令的时候,过年之时也会摆摊卖烟花爆竹什么的,从小算账,虽然帐都比较小,但也能够算清楚账目。\par
高中之时也做过一些社会实践,协助社区工作人员巡查社区,检查安全设施,维护社区治安。\par
\subsection{知识、技能与经验}
目前我具有略高于高中生的知识水平,对于语、数、外、物、化、生、政、史、地等学科的基础知识已经掌握,对于计算机科学与技术,目前已初步掌握一些程序设计知识。\par
因从小锻炼,所以对经商,做一些小买卖具有一些经验与见解。
\subsection{兴趣爱好与特长}
我最喜欢的体育运动是乒乓球,乒乓球我虽然打得不是很好,但也从中学会了很多。\par
最喜欢的学科是数学与生物,学习他们我才能感受到学习的乐趣,才能真正体会到知识的海洋是多么的广阔。\par
上大学以来,对于程序设计我又产生了浓厚的兴趣,在其中,我深刻体会到了一种创造的感觉,编程完成的时候,我往往会体会到自己创造了一个世界一般,这种感觉是极其美妙的。\par
\section{环境分析}
\par
\subsection{社会环境分析}
计算机科学与技术就业形势分析I 业一直是国家优先发展的重点行业,也是国内外人才需求量最大的行业之一。伴随着互联网的发展,IT 人才的短缺现象将会越来越严重。在我国,IC 人才、网络存储人才、电子商务人才、信息安全人才、游戏技术人才严重短缺在软件人才层次结构上,水平高的系统分析员和有行业背景的项目策划人员偏少,同时软件蓝领也比较缺乏。据保守估计,目前中国市场对I人才的需求每年超过 20 万人,随着IT 业越来越火,各大高校计算机专业报名的人数也越来越多。而近年来,随着毕业生人数激增,就业率与供求比例明显走低。通过网络以及一些其他手段的调查发现,就计算机专业近几年的就业数据来看,主要呈现以下几个趋势:\par
1、就业率居高不下。计算机人才市场需求潜力仍然很大,计算机专业人才的市场需求具有很大的潜力,这无疑是在很大程度上为我们将来的就业提供了很大的帮助,更多的网络意见是目前的该专业大学生就业难只是一种表象,原因是大学生自身的心理定位没有调整好。\par
2、考研率持续上升,说明大学生在摆脱就业压力和个人追求方面有新的认识。从不同的角度来说,为了提高自己的专业修养以及知识储备方面来说,考研绝对是值得大家考虑的:然而也有些人认为,自己所学到的知识越多越好,获得证书越多越好,因此有些人读完硕士还要读博士,从而就在一定程度上忽略了自身其他能力的培养。综合来看,选择继续读书或是提前毕业找工作要根据个人的兴趣爱好以及自身的实际情况选取合适的定位。\par
3、热门城市就业比率下降,对计算机人才需求标准逐渐提高。根据网上调查北京、上海等大型城市近几年对计算机人才的招募情况来看,这几所城市对计算机人才的需求相对呈现饱和趋势,对毕业生的需求量也是逐渐减少。同时,其招聘标准也是逐年呈现“水涨船高”的趋势,很多企业只钟情于硕士研究生、博士生等高端人才,因此必然导致毕业生去向不佳。\par
4、毕业生选择企业方面思想日渐成熟。随着近年来三资企业用人制度的透明性,劳动价值比得不合理、淘汰现象日渐浮出水面,使得一些毕业生对三资企业持严谨态度。很多毕业生在作过程中也会对所选企业的各个方面提出质疑,这就必然导致很多人在工作过程中选择跳槽,这也充分说明了当今大学生在选择用人单位方面思想的成熟。
5、毕业生对就业的期望值有待进一步提升根据目前的市场就业反应来看,大学生再就业方面的期望值有待进一步的提升,在找工作方面还不能完全放开自己,在一定程度上受到家人及朋友各方面意见的影响,会在不知不觉中和自己的学长一类的有定工作经验的人作出比较,这就在一定程度上限制了大学生自己再就业时展示自己的机会,因此也在定程度上影响了就业形势。\par
\subsection{家庭环境分析}
婚姻状况:未婚男士。\par
经济状况:尚未实现经济独立。\par
家人期望:能够过上幸福美满的生活,找一份稳定高薪的工作,至少实现经济自由。\par
\subsection{职业环境分析}
一、计算机科学技术发展现状。\par
二十世纪九十年代以来,计算机科学技术发展体现在两种趋势上:其一是传统领域(如军事、国防、科研等)中仍然不断加深利用着计算机科学技术;还有一个就是指,现在千家万户的人们日常生活中越来越离不开计算机科学技术了。关于计算机科学技术的发展现状分析如下:\par
(一)计算机科学技术广泛应用到人们的生活当中对于目前信息化的生活时代来讲,当今社会中必不可少的一项技术就是计算机科学技术。通过计算机的网络系统,千家万户中的人们就算不出门也可以了解外面世界的各种信息了。更加方便的是只需要在家通过计算机的互联网就可以买到自己喜欢的各种商品,还可以在计算机通信上面聊天,在线授课等等,就这样通过计算机网络技术来完成各种信息的交流。行政机关单位收集各种信息也可以通过建立网站来完成,各行业之间的数据交流也可以通过计算机网络技术来完成的,人们还可以通过计算机来处理大量的数据和资料,提高工作效率,通过计算机科学技术,人们的生活与工作已经逐渐改变了。\par
(二)计算机科学技术取得了突破性进展由于科技的不断进步,促使了计算机科学技术取得了突破性的进展,这些主要表现在两个方面:第一体现在微处理器方面的进展。微处理器可以通过缩小处理器芯片内晶体管的尺寸和线宽来极大地提高计算机的性能。实际就是使用更先进的光刻技术在硅片上更加精巧地制作其上面刻饰的晶体管,而且晶体管上面连接的导线也被制作得更加细小了,这样就自然缩小了处理器内晶体管的尺寸和线宽。第二是指在纳米电子技术方面的发展。纳米技术的诞生是由于计算机技术的发展让计算机更加趋于微型化、智能化、高速化了。这是一种全新看待问题的方式,不仅是单单地变小尺寸,可以说纳米技术是革命性的技术,有效地帮助到处理计算机技术的集成度和处理速度的双重制约问题。\par
二、计算机科学技术发展前景\par
(一)立足当下,面对未来\par
计算机科学技术与工业的结合,结果就诞生了智能一体化生产;计算机科学技术和农业的结合,就产生了数字化现代农业科技;计算机科学技术和金融业的结合,使得网上银行等新兴金融机构诞生了,随之“乔布斯之问”也将因为计算机科学技术的进一步发展而得到解答。计算机科学技术会通过更加多元碎片的形式融入到我们的生活中,比如:计算机科学技术凭借自己得天独厚的优势影响及改革着教育方面的发展。\par
(二)在智能化方面上计算机科学技术的发展趋势\par
网络技术的发展是计算机技术发展的重心,直接影响着计算机的功能普及,这是因为计算机目前最主要的功能就是上网,那么在以后网络技术必定会在计算机技术中体现得更好。现在人们为了完善自身,拓展视野,从而不断通过网络从各个渠道获取对自己有用的信息。科学技术进步体现在:人们在生活中遇到的很多问题都可以利用计算机的智能化来解决,这也是文明进步的代表,这促使着计算机科学技术必须要发展到更高和更深入的层面上。\par
(三)计算机科学技术的专业化和自身运算处理能力引领科技跨越式进步\par
其他科学技术和计算机科学技术之间都是相互作用着的。其他科学技术(半导体技术、光电接口技术、微型集成电路技术、生物电子科技、软件工程等)推进着计算机科学技术快速发展,同时计算机科学技术的发展也带动着其他科学技术迈向更长远的未来。在大数据时代背景下,人们通过利用计算机分析处理能力和储备能力来更好地拓展人类的分析能力和处理自己所面临的问题。概括下来,在现代中,大众生活很明显已经离不开计算机科学技术了,而计算机科学技术的历史使命就是可以不断满足人们日益增长的物质文化需求。相信在未来广大人民的利益和国家的综合实力都会随着计算机科学技术的发展而不断提升的。\par
\subsection{地域与人际环境分析}
杭州,按行星风系的气候分带,这里属副热带高压带控制的干旱区,但由于海陆差异及青藏高原隆起所导致的温压场的改变,季风环流改变了近地面层行星风系的环流系统,变干旱的大陆性气候为湿润的亚热带季风气候,变干旱的荒漠景观为湿润的常绿阔叶林景观。再经过几千年的人为作用,使本区成为我国人口密集、经济发达的区域。\par
杭州是国家信息化试点城市、电子商务试点城市、电子政务试点城市、数字电视试点城市和国家软件产业化基地、集成电路设计产业化基地。杭州致力于打造“滨江天堂硅谷”,以信息和新型医药、环保、新材料为主导的高新技术产业发展势头良好,已成为杭州的一大特色和优势。 通讯、软件、集成电路、数字电视、动漫、网络游戏等六条“产业链”正在做大做强,有12家企业进入全国“百强软件企业”行列,15家企业进入国家重点软件企业行列,14家IT企业在境内外上市。\par

\section{职业定位}
\par
\subsection{行业领域定位与理由}
软件工程领域。\par
软件工程专业是2002年国家教育部新增专业,随着计算机应用领域的不断扩大及中国经济建设的不断发展,软件工程专业将成为一个新的热门专业。\par
软件工程专业以计算机科学与技术学科为基础,强调软件开发的工程性,使学生在掌握计算机科学与技术方面知识和技能的基础上熟练掌握从事软件需求分析、软件设计、软件测试、软件维护和软件项目管理等工作所必需的基础知识、基本方法和基本技能,突出对学生专业知识和专业技能的培养,培养能够从事软件开发、测试、维护和软件项目管理的高级专门人才。\par
中国的软件行业规模不是很大,有些软件企业在软件制作上,也只是采用了一些软件工程的思想,距离大规模的工业化大生产比较还是有一定的差距;原因有管理体制的问题,市场问题,政策问题,也有软件工程理论不全面和不完善的问题。所以软件工程的研究和应用,以及中国软件行业的进一步发展,都需要一定的既有软件工程的理论基础和研究能力,又有一定的实践经验的软件工程科学技术人员来推动。软件工程的前途是光明的。\par
软件服务外包属于智力人才密集型现代服务业。大量著名外包企业落户宁波。主要就业去向包括软件外包与服务企业、信息产品与服务企业,担任程序员、软件测试员、项目经理等工作岗位。\par
\subsection{职业岗位起点定位与理由}
程序员。\par
我没有背景,也不比别人突出,从底层做起不仅是客观现实条件,这也有利于我进一步了解这个行业的工作内容,工作程序以及很多细节末微之处。
\subsection{职业目标与可行性分析}
\par
\begin{enumerate}[(1)]
	\item 短期目标(大学4年)\par
努力学习好专业知识,熟练掌握各种程序设计知识与技巧,增强自己的专业能力。\par
	\item 中长期目标(5-10年)\par
尽可能地向上走,所谓人往高处走,只有走得更高,才能接触更先进的事物,才能有更多的机会在行业里立稳脚跟,才能得到自己想要的很多东西。\par
    \item 可行性分析\par
我性格比较坚毅,能吃苦,脑子也不笨,我觉得我坚持下去是可以实现的。\par  
\end{enumerate}

\section{实施方案}
\par
\begin{enumerate}[1、]
	\item 如何利用现有条件和自身优势以实现职业生涯目标。\par
我的所学知识中,数学算是比较优秀的,而数学是计算机学科所必备的基础知识之一,但我的数学与他人相比还是并不突出的,所以仍需努力\par
	\item 如何克服缺点、弥补不足、增长知识、提高能力以实现职业生涯目标。\par
首先要积极锻炼身体,提高身体素质,其次要多仔细研究改革学习之法,让学习更有效率。\par
	\item 如何处理人际关系和发展人脉以实现职业生涯目标。\par
	1、帮助他人成功	\par
	社交的本质就是不断用各种形式帮助其它人成功。共享出你的知识与资源、时间与精力、朋友与关系、同情与关爱,从而持续的为他人提供价值,同时提高自己的价值。\par	
	2、努力让自己的付出多于回报\par	
	因为你会为别人提供价值,别人才会联系你。所以多考虑别人而不是自己。\par
	3、不要保留	\par
	不要以为友谊是有限的。这是投资,会越滚越多。\par
	4、成功的关键是慷慨大方\par	
	在社交中通行的不是贪图便利,而是慷慨大方。\par	
	5、为发展人际关系设定计划\par	
	打造交际网络是有过程的,你的计划应当包括以下三份:\par	
	①你3年的目标,及每3个月的进度。\par	
	②列出可以帮你实现每个目标的人。\par
	③如何与第2点中列出的人联系。\par
	一但你设立了目标,就贴在你经常看的到的地方。\par
	6、在你需要前,打造好人际网络\par
	要你发现要用到别人之前,就尽早的保持联系。重要的是把这些人当做是朋友,而不是潜在的客户。\par
	7、与你认识的人保持好联系\par
	刚开始时,要关注于你当前人际网络中的人。\par
	8、大胆\par
	勇气非常神奇。才能相似的两个人发展不同,可能仅仅因为脸皮的厚薄不同。\par
	9、乐于求助于别人\par
	乐于索取可以创造出机遇。你应当像乐于帮助别人一样,乐于向他人索取。记住,要做好别人说”不”的最坏打算。\par
	10、在畏惧面前考虑到收获\par
	厚着脸皮和不认识的人说话,自然会害怕你可以失败的窘境。这是件有风险的事,要有收获,有成就就必须冒风险。而什么也不做就只有平庸。	\par
	11、尊重别人	\par
	尊重每个人,不分高低贵贱。\par
	\item 如何处理工作与家庭、生活的关系以实现职业生涯目标。\par
    在如今这个社会,家庭、事业和幸福一直围绕在我们身边,我们大多的时候在茫然地处理这三者间的关系。不同的人有不同的认知,而几乎每一个都想拥有一个幸福的家庭和快乐的生活,除此之外还要有一份稳定的工作。这就让我们不得不深思一下这三者间的关系以及怎么处理这三者间的关系。\par
	首先我们要了解到家是什么?家是我们的情感栖息地,是我们的物质后盾,是我们的娱乐天地,是我们今后发展的大本营。那么工作呢?通常我们理解为工作是一种赚钱、养家或赢得社会地位的手段。我们不应该仅仅局限在这样的思维下,还应该是身心健康的保障和展示德才的舞台,也可以是我们快乐源泉。因为不同的我们有不同的兴趣爱好,只有工作才能给我们展示的和发展的舞台。\par
	随着社会的进步、科技的发达,越来越多的人更加注重事业、金钱和享乐,而把家庭放在其后,但在他们获取成功后,在家庭方面却背上了沉重的包袱,或是忙于离婚,或是忙于应付来自于家庭内部的烦恼。可见,工作、家庭和幸福之间的影响是深刻而全面的。工作、家庭和幸福之间关系及其平衡计划已成为我们首应解决的问题。当我们没有处理好时,矛盾和烦恼常常伴随在我们左右。因此,弄清工作、家庭和幸福间相互影响的机制,深入探讨工作、家庭和幸福之间关系的理论,并积极构建工作、家庭和幸福的平衡计划,对促进组织发展和个人发展都具有十分重要的意义。\par
	在我看来工作、家庭和幸福三者间的关系相互联系相互制约的,事业是家庭的和幸福的支柱;家庭是事业和幸福的根基;幸福是家庭和事业的保证,它们之间相辅相成的,不能孤立起来看待。从管理的角度而言,工作、家庭和幸福平衡计划是帮助我们认识和正确处理家庭、工作、幸福间的关系,调和工作、家庭和幸福之间的矛盾,缓解由于工作、家庭和幸福关系失衡而给我们造成压力的计划。工作、家庭和幸福平衡计划的目的在于帮助我们找到工作和家庭和幸福需要的平衡点。而要达到这一目的,必须了解我们职业生涯不同阶段的特点以及家庭各阶段的需要、工作情境对家庭生活的影响,然后制定适当的计划。\par
	\item 如何处理释放工作压力、保证身心健康以实现职业生涯目标。\par
	多看书,出去游玩,多听音乐,少干游戏熬夜这等伤身之事。\par
\end{enumerate}
\par 

\section{评估与调整}
\par 
\subsection{评估时间}
每学年一次。\par
\subsection{评估内容}
成果目标、经济目标、能力目标、职务目标等方面,确定哪些目标已按预期实现,哪些目标商未达到,对已实现的成果总结经验,对未完成的目标分析原因。\par
\subsection{调整原则}
考虑与自身情况的匹配性、与环境的适应性、操作实施的可行性等。\par




\end{document}
