\documentclass{article}
\usepackage[UTF8]{ctex}
\usepackage{geometry}
\usepackage{natbib}
\usepackage{float}
\geometry{left=3.18cm,right=3.18cm,top=2.54cm,bottom=2.54cm}
\usepackage{graphicx}
\pagestyle{plain}	
\usepackage{setspace}
\usepackage{caption2}
\usepackage{datetime} %日期
\renewcommand{\today}{\number\year 年 \number\month 月 \number\day 日}
\renewcommand{\captionlabelfont}{\small}
\renewcommand{\captionfont}{\small}
\begin{document}

\begin{figure}
    \centering
    \includegraphics[width=8cm]{upc.png}

    \label{figupc}
\end{figure}

	\begin{center}
		\quad \\
		\quad \\
		\heiti \fontsize{45}{17} \quad \quad \quad 
		\vskip 1.5cm
		\heiti \zihao{2} 《计算科学导论》课程总结报告
	\end{center}
	\vskip 2.0cm
		
	\begin{quotation}
% 	\begin{center}
		\doublespacing
		
        \zihao{4}\par\setlength\parindent{7em}
		\quad 

		学生姓名:\underline{\qquad  马青 \qquad \qquad}

		学\hspace{0.61cm} 号:\underline{\qquad 1907010223\qquad}
		
		专业班级:\underline{\qquad 计科1902 \qquad  }
		
        学\hspace{0.61cm} 院:\underline{计算机科学与技术学院}
% 	\end{center}
		\vskip 2cm
		\centering
		\begin{table}[h]
            \centering 
            \zihao{4}
            \begin{tabular}{|c|c|c|c|c|c|c|}
            % 这里的rl 与表格对应可以看到,姓名是r,右对齐的;学号是l,左对齐的;若想居中,使用c关键字。
                \hline
                课程认识 & 问题思 考 & 格式规范  & IT工具  & Latex附加  & 总分 & 评阅教师 \\
                30\% & 30\% & 20\% & 20\% & 10\% &  &  \\
                \hline
                 & & & & & &\\
                & & & & & &\\
                \hline
            \end{tabular}
        \end{table}
		\vskip 2cm
		\today
	\end{quotation}

\thispagestyle{empty}
\newpage
\setcounter{page}{1}
% 在这之前是封面,在这之后是正文
\section{引言}
现代计算机对人类的生产生活产生了巨大的影响,其作用日益突出,在各行各业都能见到计算机的工作的身影。本文主要介绍了我对计算科学导论这门课程的一些认识与体会以及对分组演讲课题的一些问题进行进一步的说明与解释。\par

\section{对计算科学导论这门课程的认识、体会}
计算机导论课为我们学习计算机专业的新生提供了关于计算机学科的入门介绍。当我们对于学习计算机迷茫不知所措时,计算机导论课为我们指明了方向。随着科学技术的进步与发展,人类各种知识的爆炸式增长,信息时代的得到了巨大进步,我们不得不承认计算机已经成为我们人类社会的发展的重要基石,悄无声息的渗透到我们生产生活的各个方面,巨大的影响推动我们人类社会的持续发展。人类社会已经无法离开计算机,我们也应该对其有一定的了解与认识。\par
首先,《计算机导论》为我们系统全面介绍了计算机的基本知识,硬件系统,软件开发,数据库基础,计算机网络与安全,计算机的应用技术以及计算机文化等。最初,他向我们介绍了计算机的应用发展史,让我们知道了计算机从最简单的计算需求到后来的文字处理,再到高级的人工智能等,几乎服务于各个行业。接着,它介绍了许多计算机领域的许多专有名词,展示了计算机的特点与其魅力。\par
对于计算学科来说,即使某个人可以熟练地操作计算机,甚至可以进行较为复杂的程序设计,仍不能说已深入了解计算学科。正如对一个厨师而言,即使他会做蛋炒饭,会做番茄炒鸡蛋,未必有能力做出满汉全席,所以不能称得上是大厨。\par
计算科学导论让我切身体会到了计算科学知识海洋的广阔与魅力,领略到了它在生活中的独特风采。 而在计算机科学导论课程中,我了解学习了以下内容:\par

一、计算机基础知识:\par
它给我们详细地介绍了计算机的产生过程,让我们知道了计算机的的历史,让我们了解了很多为计算机发展到今天所做出卓越贡献的科学家,以及第一台数字电子计算机的诞生。然后,他详细地介绍了计算机的发展过程,从1946年第一台计算机的问世到今天,计算机一个经历了4各阶段。接着,我们了解了计算机的特点,应用领域以及发展趋势,告诉我们计算机对人们生活的影响。  然后,我们学习计算机运算基础,我们学会了数制间的转换,明白数值数据在计算机中的表示。紧接着,他告诉了我们计算机系统的组成级工作原理。让我们更深的了解计算机。\par
\begin{figure}[h!]
\centering
\includegraphics[scale=1]{mq}
\caption{第一台电子计算机}
\label{fig:mq}
\end{figure}
二、计算机的硬件系统和操作系统:\par
计算机导论简单地向我们介绍计算机硬件系统的组成、系统总线和如何组装微型计算机。然后又教我们操作系统的概念、组成、功能以及几种典型的操作系统。让我们深刻地领会到计算机的强大和不可替代的作用。\par
三、计算机软件的开发:\par
他介绍了与计算机软件开发技术有关的基本知识,如程序设计、算法与数据结构以及软件工程等。让我们能够用计算机去工作,去开发新的软件,去发展计算机的伟大影响,改变人们的生活,促进科技的发展。\par
四、计算机数据库基础:\par
他向我们简述了数据库的基本概念、数据库体系结构、数据模型和关系数据库。数据库技术是计算机科学技术中发展最快、应用做广泛的领域之一,它是计算机信息系统与应用程序的核心技术和重要基础。经过了30余年的发展,其应用已遍及各个领域,成为21实际信息化社会的核心技术之一。数据库系统包括数据库系统的定义、发展、类型、结构,以及数据库管理系统和数据库语言等基础知识。\par
五、计算机网络与安全:\par
他给我们简单介绍了计算机网络的产生与发展,以及计算机网络的功能和计算机网络的组成。随着计算机网络的发展,计算机领域的安全问题显得越来越重要。近年来,随着计算机网络的普及与发展,我们的生活和工作都越来越依赖于网络。国家政府机构、各企事业单位不仅建立了自己的局域网系统,而且通过各种方式与互联网相连。但是,我们不得不注意到,网络虽然功能强大,也有其脆弱易受到攻击的一面。所以,我们在利用网络的同时,也应该关注网络安全问题,加强网络安全防范,防止网络的侵害,让网络更好的为人们服务。\par
六、计算机的发展:\par
最后就是计算机的发展。从其发展史来看计算机必然遵循着运算速度快、计算精度高、存储功能强、具有逻辑判断能力、及具有自动运行能力这五大特点向着巨型化、微型化、多媒体化、网络化、智能化五大方向或单一或结合式的发展。这五大发展方向确立必然是依据社会各阶层的需求,而值得注意的是对于现代的冯·诺依曼式计算机存在着相当多无法解决但又极其重要的的问题。所以非冯·诺依曼结构模式计算机就有了相当大的发展必要,因此出现了光子计算机、生物计算机、量子计算机,它们都具有着无与伦比的传播速度、超级大规模的储存能力,这是极其可喜的,但功能强大必然有着极其巨大的技术开发难题。可是我们相信这些技术难题必然因人类思想的能动性得以解决。但最后还要补充一句:计算机无论怎样发展都不可能具备人的思维,计算机的存在、发展必然依据人的存在、发展,所以无论怎样发展我们人类自己才是最根本的要事。在生产中,各种机械与计算机想结合,出现了自动化、智能化、高性能,这些保证了机械工作的性能、效率,使其创造了更大效益。在生产各种零件时,特别是高精密要求的,计算机就必不可少。计算机的引入,及其智能的操作、控制将企业从生产到管理,从工人的效率到产品的精密各方面都产生了极大效益。综观世界机械工业的发展,机械与计算机的结合 、发展程度已经成为衡量一个国家工业的重要标志。\par
\begin{figure}[h!]
	\centering
	\includegraphics[scale=1]{m}
	\caption{计算机与互联网联系全球}
	\label{fig:m}
\end{figure}
\subsection{第三个子标题}
\par
我阅读了《算法导论》\citep{Thomas},引发了我对计算机科学技术的兴趣,于是阅读了《深入理解计算机之道》\citep{liujiang},基于对计算科学与技术的一些认识,我又学习了《现代操作系统的思考》\citep{gu},来探索计算科学落实在生活生产领域的实际意义。\par

\section{进一步的思考}
\begin{itemize}
    \item什么是高通量测序?\par
    基因是遗传的物质基础,是DNA或RNA分子上具有遗传信息的特定核苷酸序列。基因通过复制把遗传信息传递给下一代,使后代出现与亲代相似的性状。人类大约有几万个基因,储存着生命孕育、生长、凋亡过程的全部信息,通过复制、表达、修复,完成生命繁衍、细胞分裂和蛋白质合成等重要生理过程。生物 体的生、长、病、老、死等一切生命现象都与基因有 关。基因测序是解读生命的一种途径\citep{zhang}。\par
    高通量测序技术(High-Throughput Sequencing)又称“下一代”测序技术(Next-Generation Sequencing Technology),是对传统测序一次革命性的改变,以能一次并行对几十万到几百万条DNA分子进行序列测定和一般读长较短等为标志。同时高通量测序使得对一个物种的转录组和基因组进行细致 全貌的分析成为可能,所以又被称为深度测序(Deep高通量测序技术(High-Throughput Sequencing)又称“下一代”测序技术(Next-Generation Sequencing Technology),是对传统测序一次革命性的改变,以能一次并行对几十万到几百万条DNA分子进行序列测定和一般读长较短等为标志。同时高通量测序使得对一个物种的转录组和基因组进行细致 全貌的分析成为可能,所以又被称为深度测序(DeepSequencing)\citep{jiyin}。\par
    \item为什么要不断研究高通量测序技术,令其不断进步?\par
    随着新一代高通量测序技术的发展,每天会产生TB甚至更多的序列数据。合理诠释这些大规模及复杂高维度的数据成为获取数据后一个更大的难点,是当前生物研究的关键步骤,具有巨大的现实意义。海量高通量测序数据的存储、处理和分析都极大地挑战着当前的计算机系统和计算模式。本文将结合调研情况,尤其是华大基因的实例调研,讨论当前高通量测序数据分析的现状、问题和多方采取的措施。然而,面对高通量测序数据带来的挑战,仍需要多方密切合作和长久深入的研究。\par
    基因组数据呈指数增长,获取开销日渐低廉。高通量数据的累积需来越迫切,NCBI在200 7年推出了SRA(Sequence Read Achive)数据库,用于存储、显示、提取和分析高通量测序数据。随着基因研究技术进步,海量的数据源源不断的产生,生物信息数据的存储计算需求每12到18个月就会增长10倍,远远高于Moore定律提供的参考数值,见图3。以至于美国国家生物技术信息中心(NCBI)不得不在2011年2月关闭了SRA数据库,停止接受用户提交的下一代测序数据。然而,据阿岗实验室的Rob Edwards预测[3],目前已测序的相比于待测序的仅是冰山一角,如图4。
    
    
    
    \par
    
    
    \begin{figure}[h!]
    	\centering
    	\includegraphics[scale=1]{rrr}
    	\caption{存储与DNA测序成本对照}
    	\label{fig:rrr}
    \end{figure}


    \begin{figure}[h!]
    \centering
    \includegraphics[scale=1]{ttt}
    \caption{Rob Edwards在2007年根据已测序情况对待测序历程的预测}
    \label{fig:ttt}
    \end{figure}
    
    
    
    \item高通量测序数据分析现状。\par
    (1)软件选择难。对应某一功能有上百种软件可选,随着仪器的更新换代,数据格式的变化,同一款软件的算法不断升级。\par
    (2)分析效率不高。多为领域专家依赖脚本语言和库写成的软件,未考虑与硬件资源使用的匹配。基本少有优化,并行化,串行或多线程软件居多。\par
    (3)分析流程中多软件衔接难。多数的高通量测序数据分析需几个软件配合完成,各软件通过脚本和大数据的重复读写(数据格式也需匹配)来协调。例如,比对之后做SNP检测,那么比对结果将作为SNP分析的输入。\par
    (4)各软件资源使用特征差异大。例如,拼接软件需要大量的内存消耗,比对则是典型的数据密集计算密集。\par (5)除了各分析算法上的不断优化,当前业界突出的两方面进展表现在工作流系统和云计算的应用。比如UCSC开发的针对第二代测序数据分析的应用系统Galaxy,Notre Dame 大学仿makefile开发的用来在集群、云和网格中执行大而复杂任务的工作流引擎Makeflow;计算大规模RNA-seq数据集基因差异表达的云计算工具Myrna,基于序列片段数据进行SN Pcalling的MapReduce软件Crossbow。\par
    \item总结与讨论。\par
    随着高通量测序技术的发展,序列数据的增长势如潮水。单个实验室甚至可年产PT级数据量,如此大规模数据的有效存储、高效分析、共享再利用,都是个巨大的难题,对高性能计算系统提出了严峻的挑战。目前已测序的仍不及冰山一角,已测序中完成深度分析的少之又少,可见任重道远。在算法优化、软件并行化、流程自动化、大规模数据存储、处理及深度分析等层面,有广泛的工作需要开展。\par
    针对新一代测序数据量大、数据处理过程复杂、对计算资源要求高等特点,云计算提供了一种有效的缓解途径。云架构下的平台搭建,存储、计算软件开发,工作流框架不断完善,并发挥一定作用。在我国,华大基因是一个典型的例子\citep{mess}。\par
    但“云”不是万能的,对计算和底层硬件分布的控制能力较低,另外将大量数据在云上传递需要时间和成本。将数据集存放在公开访问的服务器上以及存储关于人类研究的数据存在隐私担忧。随着研究的深入,仍需不断深入探讨。归根到底,大数据对大系统的挑战需要存储、管理、传输、调度和计算分析全面协调,需要生物领域、计算机领域、数据统计分析等多方密切配合,长久积累深入,针对高通量测序数据及其分析使用特点,才能开发出更高效实用的系统模式。\par
\end{itemize}


\section{总结}
《计算科学导论》这门课程让我对计算科学有了比较全面的了解,增强了我对计算科学行业的兴趣,其知识海洋的广阔与神秘,激发了我不断探索的兴趣。\par
最后,在学习了《计算科学导论》后,我认为对于学习计算机我们需要做到:  
一、我们首先要学好基础。对文字、表格等的处理都是计算机课程的基础,需要一定的操作桌面的知识和能力,需要一定的工具操作能力,并且要学会PPT的制作和Matable应用程序等基础软件的操作,这都将对我们今后的发展有巨大的作用。\par
二、我们还要具有创新思维能力。计算机的操作与应用不仅需要我们要学会必要地操作技术,而且我们还要有自己的创新思维,比如当我们遇到实际问题时,我们可以编程序来解决工作中遇到的实际最优解问题,这样比较的方便、快捷。\par


\section{附录}
\begin{itemize}
    \item 申请Github账户,给出个人网址和个人网站截图
    \begin{figure}[H]
    	\centering
    	\includegraphics[scale=0.1]{maqing}
    	\caption{Github:https://github.com/maqing2001}
    	\label{fig:maqing}
    \end{figure}
     
    \item 注册观察者、学习强国、哔哩哔哩APP,给出对应的截图
     \begin{figure}[H]
    	\centering
    	\includegraphics[scale=0.1]{uuu}
    	\caption{哔哩哔哩}
    	\label{fig:uuu}
    \end{figure}
      \begin{figure}[H]
     	\centering
     	\includegraphics[scale=0.1]{jjj}
     	\caption{学习强国}
     	\label{fig:jjj}
     \end{figure} 
  \begin{figure}[H]
 	\centering
 	\includegraphics[scale=0.1]{kkk}
 	\caption{观察者}
 	\label{fig:rrr}
 \end{figure}
    \item 注册CSDN、博客园账户,给出个人网址和个人网站截图
     \begin{figure}[H]
     	\centering
     	\includegraphics[scale=0.1]{ppp}
     	\caption{存储与DNA测序成本对照}
     	\label{fig:rrr}
     \end{figure}
 \begin{figure}[H]
 	\centering
 	\includegraphics[scale=0.1]{ma}
 	\caption{博客园:https://ing.cnblogs.com/}
 	\label{fig:ma}
 \end{figure}
    \item 注册小木虫账户,给出个人网址和个人网站截图
     \begin{figure}[H]
     	\centering
     	\includegraphics[scale=0.2]{qing}
     	\caption{小木虫:http://muchong.com/bbs/space.php?uid=20372710}
     	\label{fig:qing}
     \end{figure}
\end{itemize}


\hspace*{\fill} \\

{\bf 注意,参考文献至少五篇,其中至少两篇为英文文献,参考文献必须在正文中有引用。}
\bibliographystyle{unsrt}
\bibliography{references}


\end{document}
